%\documentclass[ignorenonframetext,red]{beamer}
%\documentclass[ignorenonframetext,red,handout]{beamer}\usepackage{pgfpages}\pgfpagesuselayout{4 on 1}[letterpaper, landscape, border shrink=5mm]
\documentclass{article}\usepackage{beamerarticle}
\usepackage{listings}
\usepackage{hyperref}

\mode<article>{\usepackage{fullpage}}

% code formatting options
\lstdefinestyle{numbers}{ %
  numbers=left, %
  stepnumber=1, %
  numberstyle=\small, %
  numbersep=10pt, %
  showspaces=false, %
  showstringspaces=false, %
  showtabs=false, %
}
\lstdefinestyle{python}{ %
  language=Python, %
  style=numbers, %
  keywordstyle=\color{olive}, %
  commentstyle=\color{red}, %
  stringstyle=\color{teal}
}
\lstdefinestyle{csharp}{ %
  language=[Sharp]C, %
  style=numbers, %
  keywordstyle=\color{blue}, %
  commentstyle=\color{olive}, %
  stringstyle=\color{purple}
}

\title{Functional Programming Concepts}
\author{Philip Hanson}
\institute{Jibunu LLC}
\date{November 8, 2012}

\AtBeginSection[]%
{
\begin{frame}<beamer>
	\tableofcontents[currentsection]
\end{frame}
}

\AtBeginSubsection[]%
{
\begin{frame}<beamer>
	\tableofcontents[currentsection,currentsubsection]
\end{frame}
}

\begin{document}

\begin{frame}
\maketitle
\end{frame}

\section{Concepts}

\begin{frame}
Before diving into examples that use a functional style, let's establish what \textit{functional} means, and examine the core concepts of functional programming.
\end{frame}%
As the name implies, functional programming centers around \textit{functions} and their transformations of inputs into outputs. This can be seen in the way that a functional style tends to decompose a given problem.

\subsection{Decomposition Paradigms}
\begin{frame}
Programming languages support problem decomposition in several ways:
\begin{itemize}
\pause \item procedural (C, Pascal, BASH, Batch)
\pause \item declarative (SQL)
\pause \item object-oriented (Smalltalk, C++, Java, C\#)
\pause \item functional (Scheme, Haskell, ML)
\end{itemize}
\end{frame}

\begin{frame}
\noindent There are important differences in appearance and functionality when programming in different paradigms.
\pause
\begin{block}{Object-Oriented}
``Objects are little capsules containing some internal state along with a collection of method calls that let you modify this state, and programs consist of making the right set of state changes.''
\end{block}
\pause
\begin{block}{Functional}
``Functional programming wants to avoid state changes as much as possible and works with data flowing between functions.''
\end{block}
\end{frame}

\subsection{Functional Design}
\begin{frame}
\noindent Why is functional programming useful?
\begin{itemize}
\pause \item Formal provability
\pause \item Modularity
\pause \item Composability
\pause \item Ease of debugging and testing
\end{itemize}
\end{frame}

\noindent The above are useful to varying degrees. Formal provability, for example, is a great idea, but trying to prove anything non-trivial involves a lot of work---so it's not very practical. A more practical benefit of functional programming is that it forces you to break your problem into small pieces, resulting in more modular programs.  It’s easier to specify and write (and test!) a small function that does one thing than a large function that performs a complicated transformation.

Debugging is simplified because functions are generally small and clearly specified. When a program doesn’t work, each function is an interface point where you can check that the data are correct. You can look at the intermediate inputs and outputs to quickly isolate the function that’s responsible for a bug.

Testing is easier because each function is a potential subject for a unit test. Functions don’t depend on system state that needs to be replicated before running a test; instead you only have to synthesize the right input and then check that the output matches expectations.

\subsection{First-Class Functions and Immutability}
\begin{frame}
Functional programs are organized around \textit{functions}
\end{frame}%
instead of objects. In a functional programming language, you can hold a reference to a function---or more generally, a \textit{callable object} that accepts parameters and returns some result.

This implies the existence of some other features you may have heard of, like \textit{anonymous functions}, where the body of a function can be defined without assigning it to a specific name. The ability to work with functions in this way is referred to as making them \textit{first-class}, that is, referenceable. (It also opens the door to \textit{higher-order} functions, which are functions that accept other functions as parameters.)

\begin{frame}
Functions with no side effects are called \textit{pure functions}%
\end{frame}.
Some functional programming languages enforce functional purity unless the function is tagged to allow it to be \textit{impure}.
\begin{frame}%
Data structures are generally \textit{immutable}. An immutable object has a couple of useful properties:
\begin{itemize}
\pause \item objects cannot be changed in-place
\pause \item any operation that would change some property of an object must create a copy instead
\end{itemize}
\end{frame}
\noindent Because immutable objects cannot change, we can be sure that any reference we hold to an object will remain stable and will always contain the same information. And because pure functions have no side effects, we can guarantee that a given function call won't affect any mutable objects, either.

\section{Techniques}
\begin{frame}
Let's examine some distinctive techniques of functional programming using Python (\url{http://www.python.org}).
\end{frame}%
I've chosen to use Python for these examples because it allows us to write code in a functional style while retaining a familiar, easy-to-read syntax. Note that Python is a \textit{multi-paradigm} language rather than a purely functional language; you can write procedural programs, object-oriented programs, and functional programs all in Python.

\subsection{\tt sorted}
We have an excellent example of a pure function in the Python standard library, \texttt{sorted}. The function accepts an \texttt{iterable} and returns a sorted version of that object, leaving the original untouched.

\begin{frame}[fragile]
\begin{lstlisting}[style=python,caption={Sorting a List},label={lst:sorted}]
n = [4,8,2,6,5,9,7,1]   # original list
p = sorted(n)           # sorted copy
\end{lstlisting}
\end{frame}

\noindent If we run the code of Listing \ref{lst:sorted} in a REPL and then examine the variable values, we'll find that \texttt{n} retains its original value, while \texttt{p} contains a sorted version of the same information.

\subsection{\tt filter}
This next function shows an example of higher-order functions. As you might expect from the name, \texttt{filter} returns a filtered version of a list. But it does so using a \textit{predicate} function that is passed in along with the list to be filtered.

\begin{frame}[fragile]
\begin{lstlisting}[style=python,caption={Filtering a List},label={lst:filter1}]
def p(n):  # predicate
    return n > 3

r = [1,2,3,4,5]
m = filter(p, r)
\end{lstlisting}

\begin{lstlisting}[style=python,caption={Anonymous Functions},label={lst:filter2}]
r = [1,2,3,4,5]
m = filter(lambda x: x > 3, r)  # anonymous predicate
\end{lstlisting}
\end{frame}

Listing \ref{lst:filter1} shows a reference to the function \texttt{p} being passed as a parameter to \texttt{filter}, which returns an enumerable object. This enumerable contains all elements of the original list for which the predicate (in this case, \texttt{p}) returns \texttt{true}. Thus, the iterator \texttt{m} will produce the values \texttt{[4,5]}.

It is not even necessary to hold a reference to the predicate function: Python allows us to define anonymous functions using the \texttt{lambda} keyword. Anonymous functions can be used anywhere that a function reference is required. Listing \ref{lst:filter2} shows the use of an anonymous function in place of \texttt{p} as a parameter for \texttt{filter}.

\subsection{\tt map}
Filtering is nice (and useful), but often we want to do more with the values in an enumerable, such as \textit{transform} them. We can use another higher-order function, \texttt{map}, to apply a given transformation function to every item in a list and return the results in a new enumerable.

\begin{frame}[fragile]
\begin{lstlisting}[style=python,caption={Doubling Each Element},label={lst:map_pure}]
r = [1,2,3,4,5]
m = map(lambda x: x * 2, r)
\end{lstlisting}
\end{frame}

\noindent After executing the code of Listing \ref{lst:map_pure}, \texttt{m} would contain an enumerable of the values \texttt{[2,4,6,8,10]}. This is purest way to use \texttt{map}, functionally speaking. But there is no requirement that the first argument be a pure function---it will happily execute with an impure function like \texttt{print}!

\begin{frame}[fragile]
\begin{lstlisting}[style=python,caption={\texttt{map} With an Impure Function},label={lst:map_impure}]
r = [3,2,1,'Liftoff']
m = map(print, r)
\end{lstlisting}
\end{frame}

\begin{lstlisting}[caption={Output of Listing \ref{lst:map_impure}},label={lst:map_impure_out}]
3
2
1
Liftoff
[None, None, None, None]
\end{lstlisting}

\noindent Executing the code in Listing \ref{lst:map_impure} will print the value of each element in the list when the values of the enumeration \texttt{m} are read. Note that \texttt{m} still contains an enumerable of length 4, as shown on the last line of Listing \ref{lst:map_impure_out}, but the values are useless because \texttt{print} returns void.

\subsection{\tt reduce}
The final operation we might want to perform on a data set is to aggregate it. Perhaps the most familiar aggregation function is \texttt{sum}, which performs repeated addition on subsequent elements of a list.

\begin{frame}[fragile]
\begin{lstlisting}[style=python,caption={List Multiplication},label={lst:reduce_mult}]
from functools import reduce
r = [5,6,7,8]
reduce(lambda a,b: a*b, r)
\end{lstlisting}
\end{frame}

\noindent The result of Listing \ref{lst:reduce_mult} is 1680, which is equal to $5 * 6 * 7 * 8$. Note that \texttt{reduce} also has an optional third argument, which is the initial value of the accumulator. If the third argument to \texttt{reduce} is not supplied, the first value of the list is used as the initial value of the accumulator.

\begin{frame}[fragile]
\begin{lstlisting}[style=python,caption={Convert to Dictionary},label={lst:reduce_dict}]
from functools import reduce
def da(a,b):
    a[b[0]] = b[1]
    return a

x = [('a', 1), ('b', 2), ('c', 3)]
y = functools.reduce(da, x, {})
\end{lstlisting}
\end{frame}

\noindent After executing the code in Listing \ref{lst:reduce_dict}, the dictionary \texttt{y} will contain the values \texttt{\{'a':1, 'c':3, 'b':2\}}.

\subsection{Partial Function Application and Composition}
Partial function application, also known as \textit{currying}, is the technique of transforming a function that takes $n$ arguments into a function that takes $m$ arguments, where $m < n$, by providing $n-m$ arguments.\footnote{More correctly, it is the technique of transforming a function of $n$ arguments into a chain of $n$ functions of $1$ argument.} Using partial function application, we can use \texttt{reduce} to give a name to our list-multiplication function from Listing \ref{lst:reduce_mult}:

\begin{frame}[fragile]
\begin{lstlisting}[style=python,caption={List Multiplication},label={lst:partial_mult}]
from functools import reduce, partial
mult = partial(reduce, lambda a,b: a*b)
mult([5,6,7,8])
\end{lstlisting}
\end{frame}

\noindent Again, the result of Listing \ref{lst:partial_mult} is the integer 1680, as expected. The first argument of \texttt{reduce} is stored for later using a \textit{closure}. Closures are beyond the scope of this treatment, but they're a very useful concept and merit further reading if you're interested in the mechanics of how arguments are saved in the process of currying.

\begin{frame}[fragile]
\begin{lstlisting}[style=python,caption={Function Composition},label={lst:composition}]
from functools import reduce
r = [1,2,3,4,5,6]
f = lambda x: x > 3
m = lambda x: x / 2
d = lambda a, b: a * b
reduce(d, map(m, filter(f, r)))
\end{lstlisting}
\end{frame}

\noindent Listing \ref{lst:composition} produces a result of $15.0$, or the product of the halves of all the values in the original list greater than three. This is simple (explicit) function composition and not proper function compositiion. While the \texttt{functional} module does contain a \texttt{compose} function, it is very limited compared to the facilities available in functional languages, and C\# doesn't include any facility for proper function composition at all, so we won't cover it. Explicit function composition is very useful, though, so remember to use it!

\section{Functional Programming in C\#}
Like Python,%
\begin{frame}
C\# is a \textit{multi-paradigm} programming language%
\end{frame}.
You are probably used to using it as a procedural or object-oriented language, but there are many opportunities for using a functional style, even in the standard library. Let's look at several of the tools at our disposal and then run through some analogs to \texttt{filter}, \texttt{map}, and \texttt{reduce} as they exist in C\#.

\subsection{Immutable Objects}
Objects are mutable by default in C\#. When you make a property public and give it a getter and a setter, any code that holds a reference to that object can modify it. However, we have the tools to make objects immutable by restricting all properties to use only getters and making all fields private. This will prevent unintentional modifications to our objects.

\subsection{Anonymous Methods}
C\# provides a way for us to define methods inline and use them as anonymous functions. In C\# 2, this is possible using the \texttt{delegate} syntax:

\begin{lstlisting}[style=csharp,caption={Delegate Syntax},label={lst:delegate}]
addresses.FindAll(delegate(Address d) { return d.Number > 150; });
\end{lstlisting}

When using delegates, the type of each argument must be specified, and the method body must contain a return statement if a return value is expected. C\# 4 introduced the \textit{lambda}\footnote{A term derived from the \textit{lambda calculus}: \url{http://enwp.org/Lambda_Calculus}} syntax (\texttt{=>}), which is more flexible:

\begin{lstlisting}[style=csharp,caption={Delegate Syntax},label={lst:lambda}]
addresses.FindAll(d => d.Number > 150);
\end{lstlisting}

Where the return value is the value of the expression body implicitly. This is very convenient for typing; however, it means that an anonymous method defined using the lambda syntax \textit{must} return a value. Most of the time that is the desired behavior anyway, but you should be aware of the limitations.

\subsection{Techniques in C\#}
Now let's compare the \texttt{filter}, \texttt{map}, and \texttt{reduce} equivalents in C\#. With the introduction of the lambda syntax and the LINQ libraries in version 4, these actions can be accomplished more easily than in previous versions. The table below shows the equivalent list operation for version 2 and version 4:

\begin{frame}
\begin{center}
\begin{tabular}{l|c|c}
 & C\# 2 & C\# 4 \\
\hline
\texttt{filter} & \texttt{.FindAll()} & \texttt{.Where()} \\
\texttt{map} & \texttt{.ConvertAll()} & \texttt{.Select()} \\
\texttt{reduce} & \texttt{for} loop :( & \texttt{.Aggregate()}
\end{tabular}
\end{center}
\end{frame}

\begin{lstlisting}[style=csharp,caption={Filter in C\#},label={lst:csharp_filter}]
addresses.FindAll(delegate(Address d) { return d.Number > 150; });  // C# 2
addresses.FindAll(d => d.Number > 150);                             // C# 4
\end{lstlisting}

\begin{lstlisting}[style=csharp,caption={Map in C\#},label={lst:csharp_map}]
addresses.ConvertAll<string>(
          delegate(Address d) { return d.City + " " + d.State; });  // C# 2
addresses.Select(d => d.City + " " + d.State);                      // C# 4
\end{lstlisting}

\begin{lstlisting}[style=csharp,caption={Reduce in C\#},label={lst:csharp_reduce}]
int n = 0;
foreach(Address d in addresses)                  // C# 2
    n += d.Number;

addresses.Aggregate(0, (a, b) => a + b.Number);  // C# 4
\end{lstlisting}

\section*{Resources}
\begin{frame}
\mode<beamer>{\frametitle{\secname}}
\begin{itemize}
	\item Python Functional Programming HOWTO\\ (\url{http://docs.python.org/3/howto/functional.html})
	\item Python {\tt functools} Documentation\\ (\url{http://docs.python.org/3/library/functools.html})
	\item Lambda Expressions on MSDN\\ (\url{http://msdn.microsoft.com/en-us/library/bb397687.aspx})
	\item LINQ to Objects on MSDN\\ (\url{http://msdn.microsoft.com/en-us/library/bb397919.aspx})
\end{itemize}
\end{frame}

\end{document}
